%%%%%%%%%%%%%%%%%%%%%%%%%%%%%%%%%%%%%%%%%%%%%%%%%%%%%%%%%%%%%%%%%%%%%%%%%%%%%
% Chapter 1: Motivaci\'on y Objetivos 
%%%%%%%%%%%%%%%%%%%%%%%%%%%%%%%%%%%%%%%%%%%%%%%%%%%%%%%%%%%%%%%%%%%%%%%%%%%%%%%

El objetivo de este trabajo ha sido desarrollar un experimento, para evaluar la serie del polinomio de Taylor de una funci\'on concreta. En nuestro caso el cos(x). Para ello hemos desarrollado un programa en Python, que a partir de recibir varias cotas superiores del error (resto de Taylor) calcula el polinomio de Taylor y lo muestra por pantalla.
Tambi\'en hemos creado un programa para que nos represente en una gr\'afica los polinomios de Taylor anteriormente hallados junto con el cos(x). Y otra gr\'afica que muestra el tiempo empleado en calcular cada polinomio y mostrarlo.

%---------------------------------------------------------------------------------
