\section{Programa python para hallar el polinomio de Taylor}
\label{}

\begin{center}
\begin{footnotesize}
\begin{verbatim}

#!/usr/bin/python
import random, sys
from sympy import *
import numpy as np 
import math
import time
import timeit

Errores=[0.4789, 0.0456, 0.00005, 0.000000005, 0.000000000007]
x=2.0
c=0.0
f=(x+c)/2

valor_c = Symbol('c')
valor_a = Symbol('f')
funcion = cos(valor_c)
funcion_ = cos (valor_a)


def fac(n):
  if n == 0:
    return 1
  else:
    return n * fac(n-1)

def MostrarTaylor(c,n):
  for i in range(n + 1):
    derivada = eval(str(diff(funcion,valor_c,i)))
    if (i<1):
     sys.stdout.write((str(derivada)))
    if(i>=1):
     x=Symbol('x')
     a=(derivada*((x-c)**i))
     if (derivada != 0):
       if (derivada<0):	 
        a=-a
        sys.stdout.write((' - '+str(a) +' / '+str (i)+'!'))
       else: 
        sys.stdout.write((' + '+str(a) +' / '+str (i)+'!'))
  print '\n' 
  return 1
  
def graficaTaylor(c,n):
  v=0
  for i in range(n + 1):
    derivada = eval(str(diff(funcion,valor_c,i)))
    if (i<1):
     v=v+derivada
    if(i>=1):
     x=np.arange(-10,10,0.001)
     a=(derivada*((x-c)**i))
     if (derivada != 0):
       v=v+derivada*((x-c)**i)/fac(i) 
  return v

  
def ErrorTaylor(x,c,error):
  i=0
  derivada = eval(str(diff(funcion_,valor_a,i)))
  polinomio = ((derivada/(fac(i)))*((x - c)**i))
  while (abs(polinomio)>=error):
    i+=1
    derivada = eval(str(diff(funcion_,valor_a,i)))
    polinomio = ((derivada/(fac(i)))*((x - c)**i))
  return i

 
if __name__=='__main__':
  for error in Errores:
    n=ErrorTaylor(x,c,error)
    print ('\n %3i iteraciones para dar un error <= %.15f') %(n,error)
    MostrarTaylor(c,n-1)


\end{verbatim}
\end{footnotesize}
\end{center}

\section{Programa para representar graficamente los polinomios hallados}
\label{}

\begin{center}
\begin{footnotesize}
\begin{verbatim}
#!/usr/bin/python
#!encoding: UTF-8

#import pylab as dibujo
import sys
from sympy import *
import math
import time
import timeit

import numpy as np 
import matplotlib.pyplot as plt
import calculotaylor  #programa anterior


n=3



tiempo=[]
xtiempo=[]



graf1= plt.subplot(211)
print"Cargado el   0 por ciento de las Graficas"
i=0
for error in calculotaylor.Errores:
    start=time.time()
    i+=1
    n=calculotaylor.ErrorTaylor(calculotaylor.x,calculotaylor.c,error)
    x1=np.arange(-10,10,0.001)
    y=calculotaylor.graficaTaylor(calculotaylor.c,n)
    plt.plot(x1,y, label= 'n = %d' %(n-1))
    print"Cargado %3d por ciento de las Graficas" %(100*i/(len(calculotaylor.Errores))) # Calcula el % de los datos introducidos en el Grafica, así hace la espera mas amena.
    finish=time.time()-start
    tiempo=tiempo+[finish]
plt.plot(x1,np.cos(x1), label = 'cos(x)')    
plt.title('Series de Potencias de Taylor de grado n')
plt.legend(loc = 3) 
plt.ylim(-1.5,1.5)

for i in range (1,len(tiempo)+1):
  xtiempo=xtiempo+[i]
  
graf2=plt.subplot(212)
plt.title('Tiempo que tarda en calcular el Polinomio de Taylor')
plt.plot(xtiempo,tiempo, 'bo')
plt.xticks(xtiempo, size = 'small', color = 'b')
plt.yticks(tiempo, size = 'small', color = 'b')
plt.xlabel("Numero del error")
plt.ylabel("Tiempo")
plt.xlim(0,(len(tiempo)+1))

plt.savefig("Graficas.eps", dpi=100)
plt.show()


\end{verbatim}
\end{footnotesize}
\end{center}
